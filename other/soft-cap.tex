\section{Simple rounding algorithm for soft-capacity multiple items lot-sizing
problem with constant opening cost}

\subsection{Problem statement}

Here we are considering multiple-items lot-sizing problem with
soft capacity. The assumption we are making here is that
the ordering cost is the same across time periods.

More specifically, we have time horizon $T$ and each time period
$s$ is associated with a capacity $C_s$. There are $N$ different
items and item $i$ has demand $d_{it}$ at time period $t$. If we
use order at time $s$ to satisfy demand at $t$ for item $i$,
per unit holding cost is $h_{st}^i$. For each time period $s$,
if we open it, the opening cost is $K$ and we can at most satisfy
$C_s$ units of demand from it. However, in the soft capacity
variant, we can open a time period multiple times.


\subsection{Linear Programming forumulation}

We use decision variables $y_s$ to denote how many copies do we
open order at time period $s$ and $x_{st}^i$ to denote the fraction of
demands $d_{it}$ served by orders at $s$. Then the following is
a natural linear programming

\begin{align*}
\min \quad    & K\sum_s y_s + \sum_{i,t}\sum_{s\le t}
                d_{it}h_{st}^i x_{st}^i   \\
\text{s.t.} \quad  & x_{st}^i \le y_s          \\
              & \sum_s x_{st}^i = 1       \\
              & \sum_{i,t} d_{it}x_{st}^i \le C_s y_s \\
              & x \ge 0, y \ge 0
\end{align*}


\subsection{Rounding algorithm}

We know that if opening cost is allowed to vary according
to time period. Then the natural LP has unbounded integrality
gap. Here we provide a rounding algorithm for the case
with fixed ordering cost, which also proves the integrality
gap of LP is constant.

First, we solve the linear programming and get optimal
fractional solution $x^*, y^*$. Then we modify $x^*$
to flow-requirement $z$ in following way:
for a fixed demand $(i,t)$, we look at its fractional
assignment, we remove the last $.75$ of it and quadruple
the first $.25$ of it. Now for each requirement $z_{st}^i$,
if we can satisfy it within orders between $s$ and $t$
inclusively, then the holding cost is at most $4$ times
of the fractional holding cost.

Now we describe how to round $y$. We put all $y^*$ on the number
axis, then moving from left to right, for every interval of
length $.25$, we open the one with largest capacity. In the end,
is there is any remaining, we just discard it. Clearly,
the total opening cost is at most $4$ times of the fractional
opening cost.

Now, we just need to prove there is a feasible assignment
that satisfies requirement $z$ and capacity constraints.

Here, we run the Median Assignment Algorithm in Lordi paper.
Again, we want to show the algorithm will success. Suppose
on the contrary, the algorithm fails with some requirement
$z^i_{\tau\bar t}$. Still let $\bar r$ be the earliest period that
still has free capacity and let $F = [\tau, \bar r)$. Let $A$
be all demands within $F$ such that it has positive flow-requirement
within $F$. By similar argument, we know that all capacity
in $F$ is used to satisfy $A$ and there isn't enough. We
want to prove the opposite, that we do have enough capacity.

\begin{claim}
There is enough capacity in $F$ to satisfy requirement in $A$.
\end{claim}

\begin{proof}
First, we look at $F$, and we claim that it opened capacity
at least four times the fractional capacity within $[\tau+.25,\bar r-.25)$.
By construction of the algorithm, it at most assigns the first .25 and the
last .25 to some order outside $F$. For everything else, it opens
at least four times.

Now, take any demand $(i,t)$, because it has positive
flow-requirement within $F$, it must has at least .75 fractional
assignment within $F$. Thus, the intersection of its fractional
assignment and $[\tau+.25,\bar r-.25)$ is at least .25. This
is true for any demand $(i,t) \in A$. So take all of them aggregately,
we know they all have at least .25 fractional assignment in $[\tau+.25,\bar r-.25)$.
So the total fractional capacity in $[\tau+.25,\bar r-.25)$ is at least a quarter
of the demands in $A$. But since we opened four times the fractional capacity
in $[\tau+.25,\bar r-.25)$, we know we have enough capacity.
\end{proof}

This leads to the main result

\begin{thm}
There is a 4-approximation to the soft-capacity version of multiple items lot-sizing
problem with constant opening cost. Moreover, the natural LP has integrality gap
at most 4.
\end{thm}