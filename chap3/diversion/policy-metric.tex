\subsubsection{Metrics}

Now imagine that we are evaluating a diversion at some point of the day. The trouble is that we don't know what will happen if we divert or not divert the patient. However, let's forget about the difficulty and assume we will know what happens until end of the day for both cases: we divert or not divert. Looking at every information or outcome we'd like to know, how do we decide which outcome is more desirable. Denote the outcome if we don't divert the patient as $\delta$. Denote the outcome if we divert the patient as $\delta'$. We'll describe metrics we care about these outcomes.

\paragraph{Waiting time} Waiting time is the most important metric. We can look at when does each patient come to their site and when their scans begin. The difference will be the waiting time for each patient (however, if a patient comes earlier than his/her appointment, we don't count the time between their arrival and their appointment as waiting). Let $W(\delta), W(\delta')$ be the average waiting time respectively.

\paragraph{Overtime} Each site has operating time. If the site has to continue operate and staffs stay for work, then overtime is incurred. The reason is usually because there are patients still need processing even after the close time of the site. We don't want our diversion to incur a lot of overtime. Different sites have different operating time. Suppose one site close at 8pm and another close at 10pm. Then at 9pm, the first site seems pretty empty so we may tempt to divert patient to there. However, this is cheating since the first site is supposed to be closed. If we divert patient there, we will incur overtime for that site, which must be accounted for. More precisely, we define overtime of a site to be the time it's running outside its regular operating time. Let's use $O(\delta), O(\delta')$ to denote overtime for each outcome.

\paragraph{Waiting time of diverted patient} Let $P(\delta), P(\delta')$ be the waiting time of the diverted patient in each outcome. We care about this metric because we don't want to divert patient to site where they're going to wait longer compared to if they stick to their original site.

Overall, a desirable diversion should have two properties. First, we want $P(\delta) > P(\delta')$. In order words, we want the diverted patient to wait less for incentive reason. Second, let $C(\delta) = w_W W(\delta) + w_O O(\delta)$ be an overall cost which consider both average waiting time and overtime through weights. Then we can define $C(\delta')$ accordingly. We'd like our diversion to have $C(\delta) > C(\delta')$. In other words, we want the overall cost to decrease.