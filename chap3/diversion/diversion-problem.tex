\subsection{Problem}

\paragraph{Scan duration} Scan duration is highly variable and is one of the main contributors of uncertainty. For example, one most popular scan is BRAIN WITHOUT AND WITH CONTRAST. This involves scan patient's brain for two phases: one without contrast and the other with contrast. During the scan, a nurse will come to inject contrast to patient via intravenous therapy. Besides this, the scan consists of almost ten different image series, each with different parameters and different orientation. If the patient moves too much during any series, the images produced may become very blurred, rendering it useless for diagnosis. In this case, the technologist will have no choice but to redo this series. Each series may take minutes, so redoing one will introduce nontrivial delay to the system. On top of that, patients may have different body size, which can also impact the amount of images needed to be acquired. Patients may need to go to restroom during the scan, they may be of poor health condition and require other people to help them move onto and off scan table. All these and other factors can affect amount of time the machine is occupied and they together make the scan duration highly variable. Sometimes, patients may experience claustrophobia during scan. The amount of time needed to allow patients to get used to the narrow space inside MRI machines and calm down is very uncertain. This will increase duration this patient occupied the resource and add delay to the system.

\paragraph{Patient arrival} Patients are supposed to come at their appointment time. But this rarely happens in reality. Due to several reasons, patients may come earlier or later than their appointment time. Patients come with different transportation options. Some patients come by car. Manhattan is a place where bad traffic is very common. It's not hard to get stuck behind traffic for half a hour. Some patients come by subway or train. This is usually more predictable. However, if the patients are not familiar with these public transportation and, especially, how and where to transfer, then they may need additional time to get through the transportation. In some unusual case, the subway and trains may be stopped due to maintenance or other reasons. Delay will occur under such scenarios. Probably the most unpredictable transportation method is Access-A-Ride. Some patients have medical conditions that require them to have access to medical care through transportation. In this case, they may resort to Access-A-Ride. These vehicles have medical equipments with them. However, patients need to make appointment with these service beforehand. Before the day of transportation, the company will collect all patients who have appointment on that day and optimize to get a route that can accommodate the pickup and drop-off needs of each patient well. However, the problem with this is that if something doesn't work as expected for early patients, then the effect will propagate through downstream patients. It's quite common that the driver cannot find a patient at the right time and location due to mis-communication. As a result, patient may come to site very early or late respect to their appointment if they use such service. Patients themselves are also aware about the uncertainty associated with the transportation. As a result, they will adjust themselves by moving out of their house earlier. However, due to the difficulty of accurately estimating travel time, some patients still end up come to site at time far away from their appointment time.

\paragraph{Unexpected events} Same Day Add-on patients are emergent cases that need to be taken care of. Because it's very unexpected, they usually come to the imaging facility without appointment. Thus, the system have no knowledge about them until they show up. Because the facility is obliged to serve them, all downstream patients will be pushed off. Delay happens. Another unexpected events is cancellation. Sometimes, patients cannot come to their appointment due to various reasons. In this case, they sometimes will call the hospital to cancel their appointment or reschedule to other days. These two cases are identical to us because if the patient reschedules to another day, it's no longer our concern for today. But anyway, this may leave gap in schedule. Maybe it will cause some resources to idle. Maybe it can help the site to catch up with their schedule if they're already behind their schedule. It's not necessarily bad, but this definitely create unbalance across different sites.

\paragraph{Schedule unbalance} The facilities are capable of processing hundreds different types of MRI scans. With slightly different characteristics, each scan may display a different stochastic pattern for their durations. This will introduce some unbalance between facilities throughout the day. Another reason for unbalance is resulted from scheduling. All scans must be scheduled into slot with size a multiple of 15min. This results that some scans are scheduled tightly and some are scheduled tightly. If some facilities have several tight scans around the same time, then it's quite possible congestion will develop there.
