\subsubsection{Prediction}

It would be easy if we can foresee future. However, we cannot. So we need to make prediction about future in order to choose between alternatives. There are several uncertainties that will affect future

\begin{itemize}
\item For patients who are under scan, how much more time do they need
to finish their scans?
\item For patients who haven't shown up, when will they show up?
\item For patients who are waiting and who haven't shown up, how long
will their scans take?
\item Which patients will cancel there appointment and when?
\item Will SDAOP patients show up in future, if so, where and when?
\end{itemize}

All of these uncertainties will affect how future unfold. Different scan durations will determine which site will become overloaded and which site will become underloaded. Although we don't know their exact value, but we do have knowledge of their distribution. For patients who are already under scan, we know the conditional distribution of remaining scan time.

We can make prediction about future by sampling random variables from our distributional information. Although we also have distribution information about the last two, but we decide to not consider them. The reason is that these two are rare events and consider them will increase the variance of our simulation without providing much benefit.

One thing we want to note here is that we estimate all distributions from data, which may be different from reality. There could be bias
containing in the data and there could be shifting trends in data. But our model doesn't require that we have perfect distribution
information. As long as our knowledge provides some good approximation of the reality, we should be able to capture enough information about future to allow us to make predictions.

In order to reduce variance, we use the common random number technique. In order words, to compare two options to assign patient, we use the same set of random numbers sampled from (conditional) distribution. Because the two options share the same future uncertainty, the only difference will come from the fact that patient will go to different sites.

As a result, when evaluating a diversion, we can obtain a set of outcomes $\{\delta_k\}_{k=1}^K$ for not diverting the patient and a set of outcomes $\{\delta'_k\}_{k=1}^K$. Here we used the same random numbers to obtain outcome $\delta_k$ and $\delta'_k$. In other words, the only difference in these two outcome is the patient in question is assigned to different site. It becomes very easy to compare these two outcomes.
