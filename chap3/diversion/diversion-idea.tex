\subsection{Idea}

Since there are lots of uncertainty that we cannot plan for, we decide to take a more reactive approach. As the day evolves, we learn more and more about these uncertainties. For example, maybe several scans take longer than expected for one site or some SDAOP show up at one site. We can balance system loads by moving patients from congested facility to non-congested facility in real-time. In this way, we can smooth out system unbalance and reduce instability. Also, in most cases, the diverted patient can expect to wait a lot less since he/she is diverted from a congested site to a non-congested site.

\paragraph{Lead time} However, the challenge is that these sites are not collocated, so we need to find a way to divert patient to other sites. We overcome this by giving patient some lead time and call them to notify them about diversion opportunities in advance with respect to their appointment time. Since all sites are close to each other, we think one hour (or 90min) is a reasonable lead time. Suppose a patient has appointment at 3pm. At 2pm, we identify a beneficial diversion. We will call the patient to tell them about this. Maybe the patient is still at home so it's relatively easy for him/her to go to a different site. Maybe the patient is on his/her way and then whether it's easy for him/her to go to a different site depends on how far he/she goes.

\paragraph{Patient's choice} Of course, we cannot force patient to divert. Our phone call is just offering patients the opportunities to be diverted. Patients can accept or decline the offer. If they decline it, they will still go to their original assigned site. Patients may have some personal stuff to do around the sites before or after their scans. Or maybe their transportation doesn't have flexibility to go to a different place. We will fully respect patient's choice.

\paragraph{Incentive} It's possible that a diversion can benefit the overall system, but the patient being diverted doesn't have any benefit or even suffer from longer waiting time. This will create incentive issues and patients won't want to agree with diversions. So when considering diversions, we want to investigate the impact on diverted patient and make sure that they benefits from the diversion, i.e. themselves will experience less waiting time if they accept the diversion.

\paragraph{No surprise} It's hard for patients to accept diversion if they didn't anticipate it coming. Also, they may be unprepared to receive phone call so they may miss it. Some patients don't even the have the flexibility/interest for diversion. So it doesn't make sense to try to divert patients who don't have the flexibility (for example, patients who use Access-A-Ride for transportation). The night before their appointment day, the hospital people will call them to remind them about the forthcoming scans. At that time, they can also tell patients about this diversion and ask them whether they want to be volunteers to have the opportunity to go to another site if their original one is very congested. If patients are interested in diversion, then we mark them as volunteers. We will only consider diversions on volunteers. In this way, patients can be better prepared for diversion both physically and mentally. They may watch out for phone call from hospital one hour before their appointment. Also, they may think about ways to go to different sites beforehand. All these can help patients go to a different site more smoothly should they agree to divert. 

\subsection{Benefits}

The main benefits of diversion is to pool resources together and hopefully decrease patient waiting time. Since we're balancing loads across several sites, we may reduce the number of cases with extremely waiting time effectly.

Another benefit is that by sharing loads between facilities, it's less likely idle time will develop. Image a case where one facility is on schedule and another is behind. It's likely the one on schedule may develop some idle time from time to time because scan finished sooner than their expected time. But for the one behind schedule, they need to work non-stop until all backlogs are processed. However, in retrospection, we may have wasted some machine time in the first facility. If we can let the first facility share some load from the second one, then they will both work to catch up with backlogs, resulting in less idle time. Overall, this will reduce overall waiting time because machines are utilized more efficiently.
