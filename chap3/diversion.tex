\section{Policy}



Usually, each day starts with a nice schedule and gradually develops congestions here and there. We have already see why in last section. Here we describe our idea to deal with it as it happens. We mainly want to handle uncertainties that are hard to plan for beforehand, so we decide to take a more reactive approach. This complements good planning and further improves system performance.

As the day evolves, we learn more and more about these uncertainties. For example, maybe several scans at one site took longer than expected or some SDAOP showed up. In either case, that site is probably running behind schedule now. We can balance system loads by moving patients from that congested facility to non-congested facility in real-time. In this way, we can smooth out system unbalance and reduce instability. This helps us improve patient waiting time and reduce site overtime.

\subsection{Why this helps?}

One may concern that by moving patient from congested site A to another site B, we are just moving problems to other place and overall system doesn't run any better. This is definitely legitimate concern, but there are several reasons why diversions can still help.

We are mainly concerned with cases where patients experience extreme long waiting time. By moving patient to non-congested site, we can balance waiting time and overtime among sites thus reduce extreme cases. This in itself is considered beneficial.

Furthermore, by pooling resources, we can alleviate the problem. During daily operation, there are some machine idling time. Appointment cancellation will leave a gap in schedule and machine time may be wasted. If some scan finishes sooner relative to scheduled slot size, the machine time until next patient's appointment may be wasted. If we can somehow leverage these wasted machine time, then patients can generally wait less and sites can incur less overtime. Idle time are more likely to develop at non-congested site since congested ones have to work non-stoping to catch up. By diverting patients from congested site to non-congested one, we will use some machine idle time which will be otherwise wasted. This helps us to process more patients in same amount of time and helps patients to move faster in the system and wait less. This will also help sites to finish the day earlier.

\subsection{Protocol}

There are several issues need to be cleared out before we can implement this idea. All of them are related to the fact that these sites are not collocated. Of course, if all sites are at the same location, all resources are already perfectly pooled together and we won't have any work to do.

\paragraph{Lead time} Patient need some time to prepare to go to a different site.  So we call them about diversion opportunities in advance with respect to their appointment time. Since all sites are close to each other, we think one hour (or 90min) is a reasonable lead time. Suppose a patient has appointment at 3pm. At 2pm, we identified a beneficial diversion. We will call the patient to tell him/her about it. Maybe the patient is still at home so it's relatively easy for him/her to go to a different site. Maybe the patient is on his/her way and, hopefully, it's still easy to change destination. If not, we'll simply ignore this diversion opportunity.

\paragraph{Patient's choice} Of course, we cannot force patient to divert. Our phone call is just offering patients the opportunities to be diverted. Patients can accept or decline the offer. If they decline it, they will still go to their original assigned site. Patients may have some personal stuff to do around the sites before or after their scans. Or maybe their transportation doesn't have flexibility change destination. We will fully respect patient's choice.

\paragraph{Incentive} It's possible that a diversion can benefit overall system, but the patient being diverted doesn't have any benefit or even suffer from longer waiting time. This will create incentive issues and patients won't want to accept such diversions. So when considering diversions, we want to take into consideration the impact on diverted patient and make sure that they experience less waiting time by going to a different site.

\paragraph{No surprise} It's hard for patients to accept diversion if they didn't anticipate it coming. Also, they may be unprepared to receive phone call so they may miss it. Some patients don't even the have the flexibility/interest for diversion. So it doesn't make sense to try to divert patients who don't have the flexibility (for example, patients who use Access-A-Ride for transportation). The night before their appointment day, the hospital people will call them to remind them about the forthcoming scans. At that time, they can also tell patients about potential diversions and ask them whether they want to be volunteers to have the opportunity to go to another site if their original one is very congested. If patients are interested in diversion, then we mark them as volunteers. We will only consider diversions on volunteers. In this way, patients can be better prepared for diversion both physically and mentally. They may watch out for phone call from hospital one hour before their appointment. Also, they may think about ways to go to different sites beforehand. All these can help patients go to a different site more smoothly should opportunities arise.
