\section{System dynamics}

Our approach is to divert patients proactively to prevent
congestion to happen. We set a lead time $l$. If we identify
a beneficial diversion, we will call the patient $l$ time
before his/her appointment time and offer the patient
the opportunity to be diverted. The patient can accept or
decline this diversion. All patients are marked volunteer
or not. This information is acquired before the scan day.
Only volunteers can be diverted. To make sure patients
are incentived to cooperate, we try to pick diversions
that will make the patient to be diverted wait less.

We use discrete-event system to describe our system. We're going
to describe the entities and their associated states. Then, we
will describe all the events and how they change system states.
Finally, we will describe the inputs of this model and how do
we use data to fit these inputs.

\subsection{Entities and states}

There are multiple sites in the system:
\begin{itemize}
\item Each site has one or more machines.
\item Each machine can process at most one patient at any time.
Once a scan is started, it cannot be preempted.
\item There is a single waiting room in each site to accomodate
      patients who have arrived but haven't started the scans.
\item Each site has a open time and a closing time for the day.
      It's supposed it doesn't operate outside this time window.
\end{itemize}
We make a simplifying assumption here: all machines are homogeneous
and are capable to conduct any types of scans. In reality, this is not
true, but the difference is quite small that we think it's okay to
ignore it.

The central entity is appointment, each appointment is associated
with one patient and has the following states:
\begin{itemize}
\item Appointment time
\item Slot size
\item Appointment status. This could be
  \begin{itemize}
  \item "scheduled but hasn't shown up"
  \item "arrived and preparing for scan"
  \item "ready for scan"
  \item "under scan"
  \item "scan completed"
  \item "canceled"
  \end{itemize}
\item Assigned site. This may change due to diversion.
\item Type of scan.
\item Machine occupied. Only valid when patient is "under scan".
\item Various timestamps
  \begin{itemize}
  \item Arrival time
  \item Ready time
  \item Scan begin time
  \item Scan completion time
  \end{itemize}
\end{itemize}
For the Same Day Add-on Patient (SDAOP), they will show up at
a site without appointment. So in our system, there is no
appointment for them until their arrivals.

Last but not least, we also track the current time in the system.


\subsection{Events}

The following are all events that we consider
\begin{itemize}
\item Patient arrival. This happens when a patient arrives at
      the assigned site. Note this can be a SDAOP.
\item Patient ready. This happens when a patient finished preparation
      after arriving at the site. The preparation may include filling
      paperwork, changing clothes, being put on intravenous therapy (IV),
      etc.
\item Cancellation. This happens when a patient notify the hospital
      that he/she will not fulfill the appointment.
\item Scan begin. When a machine is idling, the patient who is ready
      with earliest appointment time will be started.
\item Scan completion.
\item Diversion. This happens when we successfully divert a patient
      to another site. 
\end{itemize}
We assume there is no no-show in numerical experiment. In practice,
no-show is quite rare since it's not easy to get an appointment. So we think it's
okay to omit it in numerical experiments. In practice, we can deal
with no-show by changing the appointment's status to "canceled"
when the patient doesn't show up by certain threshold after
the appointment time. If the patient does show up later, we will
change the status to "arrived and preparing for scan".

We also assume all diversions are successful in numerical experiment.
This is because we will experiment with different fraction of volunteers
and low fraction of volunteers can be partially interpreted as
unsuccessful diversions.

\subsection{Model inputs}

The model needs following inputs
\begin{itemize}
\item The schedule at the beginning of the day. This includes
      appointments across all sites.
\item The set of patients who are volunteers.
\item The distribution of arrival time per appointment.
\item The distribution of preparation time per appointment.
\item The distribution of scan duration per appointment.
\item The set of SDAOPs and their arrival time.
\item Cancellations.
\end{itemize}
Here we describe a ideal scenario where we know distributional
information for each appointment. In practice, we need to
use historical data to fit these distributions and we cannot
hope to reach the granularity of per appointment level.
For scan duration, we fit one distribution per type of scan.
For arrival time and preparation time, we fit one distribution
for each for the whole population. For sure, this can be further
improved with more effort.

We are making (admitted large) assumptions on these randomness.
For example, when a site is running behind schedule, staffs
may work faster. Scan duration is dependent on the current status
of a site. There are lots of factors in reality and we cannot
hope to model all of them. So we focus on constructing a model
that is qualitatively similar enough to reality.

These model inputs are used to simulate operation for one day.
We don't allow our diversion policy to peek into future, so
the policy is not aware of SDAOPs and cancellation until they
happens.

We will discuss how to fit these inputs in the section for
numerical experiments.

\subsection{Metric}

There are two things we care about: patient waiting time and
site overtime. For one patient, if the appointment time is $a$
and the scan begins at time $t$, then waiting time is defined
to be
\[  (t - a)^+ \]
For each machine, the overtime $o$ is defined to be the amount of
time this machine is working outside the operating time window
of the site where the machine resides.

We're not only interested in average waiting time/
overtime, we are interested in the distribution of waiting time
and overtime. For example, we pay lots of attention on numbers of extreme
waiting time (for example, $\ge$60min or $\ge$90min waiting time)
since they're very important for service quality perceived by patients.
As a result, we will demonstrate the impact of diversions via
multiple statistics on waiting time and overtime.
