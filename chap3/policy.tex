\section{Policy}

In this section, we're concerned with the policy we use to identify
beneficial diversions. Before going to details, we'd like to discuss
some potential benefits of diversions. First of all, we can at least
balance loads between sites and let them share the pain when congestion
happens. This alone can reduce extreme waiting time and extreme overtime.
Secondly, we can make more efficient use of machine time. We can make
use of the slot of canceled appointment or any gap in schedule.
When a site is on schedule, idle time can develop whenever a scan finishes
sooner than expected. But when a site is behind schedule, idle time is
unlike to occur. Thus, when we let sites share the pain, we can make
more efficient use of machine time across all sites.

Now we'll discuss how the details on identifying beneficial diversions.

\subsection{Objective}

At the end of the day, given a outcome $\delta$, we'd like to measure
its desirability with a objective function. Let $W(\delta)$ be the
waiting time vector (one component per patient) and $O(\delta)$ be the overtime
vector (one component per machine). We define the following objective
\[  C(\delta) = ||W(\delta)||_2 + \lambda ||O(\delta)||_2 \]
where $\lambda$ is a parameter controlling how much weight
we put on overtime compared to waiting time.

This objective captures both metrics in interest. The choice of
$l2$-norm here expresses our preference on eliminating cases of
extreme long waiting time and extreme long overtime. This will
also encourage diversions that let sites share the pain.

Another usage of this objective is to prevent the system to
divert patients to sites that are supposed to be close.
If we use $l1$-norm, this may not be achieved. Suppose Site A
closes earlier than Site B. If Site B is congested, diversion
into Site A when it's supposed to close can keep the total
overtime the same while reducing waiting time. However,
$l2$-norm can prevent this from happening (to larger extent
compared with $l1$-norm).

\subsection{Choosing diversions}

$l$ time before the appointment time of some patient, we need to
decide whether to divert this patient to another site if this
patient is a volunteer. However, the outcome is stochastic in
nature whether we divert the patient or not. Consider a diversion
$D$. Suppose without $D$ (and any future diversions), we will
reach random outcome $\delta$. With $D$ (but still no future
diversions), we will reach random outcome $\delta'$. Let
$P(\delta), P(\delta')$ be the random waiting times for the patient 
to be diverted. We say $D$ is a \textit{valid} diversion if
\begin{align*}
  p_c = \mathbb{P}[C(\delta') < C(\delta)] \ge \theta_c   \\
  p_p = \mathbb{P}[P(\delta') < P(\delta)] \ge \theta_p
\end{align*}
where $\theta_c, \theta_p$ are two treshold controlling how
aggressive/conservative we are. High threshold ensures we will
see improvement for sure, but there may only be very few
diversion options.

The critierion is to make sure a diversion will leads to
improved objective value and less waiting time for diverted
patient with sufficient probability.

It's impossible to know $\mathbb{P}[C(\delta') < C(\delta)],
\mathbb{P}[P(\delta') < P(\delta)]$. So we need to approximate
them, we decide to use simulation to estimate these probabilities.

We will generate a random vector $\xi$ includes
\begin{itemize}
\item The scan durations of all patients.
\item The arrival time of all patients.
\item The preparation time of all patients.
\end{itemize}
We sample all these from the corresponding empirical distribution,
possibly conditioning on information in hand at current time
(for example, for in-progress patient, we know how long the patient
has been under scan). We omit cancellation and SDAOP since
they're low probability event and hard to predict.

With $\xi$, we can simulate the system with/without diversion $D$
and assess the outcome. To make estimation more accurate, we will
in total general $k$ random vectors $\xi_1, \ldots, \xi_k$, each
corresponding to one scenario. And let $\delta(\xi_1), \ldots,
\delta(\xi_k)$ be outcomes without $D$ and $\delta'(\xi_1), \ldots,
\delta'(\xi_k)$ be outcomes with $D$. We use the fraction of
scenarios where $C(\delta'(\xi_i)) < C(\delta'(\xi_i))$ as
estimated objective improvement probability $\tilde p_c$. We use the fraction
of scenarios where $P(\delta'(\xi_i)) < P(\delta'(\xi_i))$ as
estimated less waiting (for diverted patient) probability $\tilde p_p$.
If $\tilde p_c \ge \theta_c$ and $\tilde p_p \ge \theta_p$, we say
this diversion is \textit{approximately valid}. Here we are
comparing pairs of outcomes coupled with same random vectors.
This technique is called Common Random Numbers and can help
reduce variance when comparing outcomes.

If there is no \textit{approximately valid diversions}, we will
keep the patient at his/her current site. Otherwise, we will
choose the diversion with highest estimated objective improvement
probability $\tilde p_c$.

We're basically greedily diverting patients. This is obviously
suboptimal. However, modeling this decision problem as Stochastic
Dynamic Programming leads to a formulation that is very hard to
solve. The simple approach we take is easy to compute and easy
to interpret. It's also extensible that we can incorporate other
desirable property in choosing diversions.
