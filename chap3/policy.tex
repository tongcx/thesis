\section{Policy}

In this section, we are concerned with the policy we use to identify
beneficial diversions. Before going into details, we will first discuss
some potential benefits of diversions. First, we can at least
balance workload between sites and let them share the pain when congestion
happens. This alone can reduce extreme waiting time and extreme overtime.
Second, we can make more efficient use of machine time. The reason for this
is that it allows us to make use of the slot of canceled appointment
or any gap in schedule;when a site is on schedule, idle
time can develop whenever a scan finishes
sooner than expected. But when a site is behind schedule, idle time is
unlikely to occur. Thus, when we let sites share the pain, we can make
more efficient use of machine time across all sites.

Now we shall discuss the details how we identify beneficial diversions.

\subsection{Objective}

At the end of the day, given a outcome $\delta$, we wish to measure
its desirability with an objective function. Let $W(\delta)$ be the
waiting time vector (one component per patient) and $O(\delta)$ be the overtime
vector (one component per machine). We define the following objective
\[  C(\delta) = ||W(\delta)||_2 + \lambda ||O(\delta)||_2, \]
where $\lambda$ is a parameter controlling how much weight
we put on overtime compared to waiting time.

This objective captures both metrics of interest. The choice of
$L_2$-norm here expresses our preference for eliminating cases of
extremely long waiting time and extremely long overtime. This will
also encourage diversions that allow sites to share the pain.

Another aim of this objective function is to prevent the system from
diverting patients to sites that are supposed to be closed.
If we use $L_1$-norm, this might not be achieved. Suppose that Site A
closes earlier than Site B. If Site B is congested, diversion
to Site A when it is supposed to be closed can keep the total
overtime the same; while reducing waiting time. However,
$L_2$-norm can prevent this from happening (to a larger extent
when compared with the $L_1$-norm).

\subsection{Choosing diversions}

For our given lead time $l$, $l$ time units before the appointment
time of each patient who is a volunteer, we need to
decide whether to divert this patient to another site. However, the
outcome is stochastic in
nature whether we divert the patient or not. Consider a diversion
$D$. Suppose without $D$ (and without any future diversions), we will
reach random outcome $\delta$. With $D$ (but still no future
diversions), we will reach random outcome $\delta'$. Let
$P(\delta), P(\delta')$ be the random waiting times for the patient 
to be diverted. We say $D$ is a \textit{valid} diversion if both
\begin{align*}
  p_c = \mathbb{P}[C(\delta') < C(\delta)] \ge \theta_c   \\
  p_p = \mathbb{P}[P(\delta') < P(\delta)] \ge \theta_p
\end{align*}
where $\theta_c, \theta_p$ are two thresholds controlling how
aggressive/conservative we are. High thresholds ensure we will
see improvement for sure, but there might be only very few
diversion options.

The critierion is to make sure a diversion will lead to an
improved objective value and less waiting time for a diverted
patient with sufficiently high probability.

It is impossible to compute $\mathbb{P}[C(\delta') < C(\delta)]$ and
$\mathbb{P}[P(\delta') < P(\delta)]$ precisely. So, we need to approximate
them, we decide to use simulation to estimate these probabilities.

We will generate a random vector $\xi$ that includes
\begin{itemize}
\item the scan durations of all patients;
\item the arrival time of all patients; and
\item the preparation time of all patients.
\end{itemize}
We sample all of these from the corresponding empirical distribution,
possibly conditioning on information in hand at current time
(for example, for an in-progress patient, we know how long the patient
has been under scan). We omit cancellation and SDAOP since
they are low probability events and hard to predict.

With $\xi$, we can simulate the system with/without diversion $D$
and assess the outcome. To make the estimation more accurate, we will,
in total, generate $k$ random vectors $\xi_1, \ldots, \xi_k$, each
corresponding to one scenario. And let $\delta(\xi_1), \ldots,
\delta(\xi_k)$ be outcomes without $D$ and $\delta'(\xi_1), \ldots,
\delta'(\xi_k)$ be outcomes with $D$. We use the fraction of
scenarios where $C(\delta'(\xi_i)) < C(\delta'(\xi_i))$ as
the estimated objective improvement probability $\tilde p_c$. We use the fraction
of scenarios where $P(\delta'(\xi_i)) < P(\delta'(\xi_i))$ as the
estimated less waiting time (for a diverted patient) probability $\tilde p_p$.
If $\tilde p_c \ge \theta_c$ and $\tilde p_p \ge \theta_p$, we say
this diversion is \textit{approximately valid}. Here, we are
comparing pairs of outcomes coupled with same random vectors.
This technique is called Common Random Numbers and can help
reduce variance when comparing outcomes.

If there is no \textit{approximately valid diversion}, we will
keep the patient at his/her current site. Otherwise, we will
choose the diversion with highest estimated objective improvement
probability $\tilde p_c$.

This approach basically greedily diverts patients. This is obviously
suboptimal. However, modeling this decision problem as Stochastic
Dynamic Programming leads to a formulation that is very hard to
solve. The simple approach we take is easy to compute and easy
to interpret. It is also extensible that we can incorporate other
desirable property in choosing diversions.
