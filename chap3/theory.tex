\section{Some theoretical results}

In the experiment, we considered two lead times. It's an interesting question
to ask whether the system's performance change continuously with respect to
lead time. Here we first give a proof on a simple setting and then discuss
where is the difficulty to extend it to our more realistic setting.

Consider the following setting. There are several sites, each with one machine.
There are $n$ patients with appointment and they will arrive punctually
according to their appointment times. Each patient has an exponentially
distributed service distribution, potentially having different mean.
Let the lead time be $l$. Every patient is volunteer and we use a
join-shortest-queue policy: $l$ time before the appointment of some patient,
we will look at in expectation which site will finish its current load
earliest and then put the patient there.

We will prove that the expected total waiting time is continuous with
respect to lead time $l$.

\begin{proof}
Fix the schedule (including all patients' appointment times, mean of their
service durations). Let $X(l)$ be the random total waiting time when we
use lead time $l$. Here the randomness come from the randomness of service
durations. We want to prove that $E[X(l)]$ is continuous w.r.t. $l$.

Now consider two lead times $l$ and $l+\epsilon$.
\end{proof}
