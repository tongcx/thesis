\section{Continuity of system performance with respect to lead time}

In the experiment, we considered two lead times. It is an interesting question
to consider whether the system's performance change continuously with respect to
lead time. Here we first give a proof on a simplified setting.

Consider the following setting.
\begin{itemize}
\item There are $k$ sites, each with one machine.
\item There are $n$ patients, numbered $1,\ldots,n$. Patient $i$
      has appointment time $a_i$ and the appointment times are
      in non-decreasing order. Patients will arrive almostly
      punctually
      according to their appointment times.
\item The service duration $d_i$ of patient $i$ follows exponential
      distribution $Exp(\mu_i)$ with mean $\frac{1}{\mu_i}$.
\end{itemize}
The system will follow join-shortest-queue policy with lead time $l$.
Let $P_j$, at any given time, be the set of patients who are assigned
to site $j$ but hasn't completed their services.
At time $a_i-l$, we will try to assign patient $i$ (for patients
with the same appointment time, we will assign patient with lower
number first). The system will look at each site $j$ and its
expected total process time
\[  T_j = \sum_{i' \in P_j} \frac{1}{\mu_{i'}}  \]
and assign patient $i$ to the site with lowest $T_j$

and the patients will arrive punctually according to their appointment times.
Each patient has an exponentially distributed service distribution $d_i \tilde Exp(\mu_i)$.
Let the lead time be $l$. Every patient is volunteer and we use a
join-shortest-queue policy: $l$ time before the $a_i$, we will check which site
will finish its current load earliest in expectation and then put the patient $i$ there.

We will prove that the expected total waiting time is continuous with
respect to lead time $l$.

\begin{proof}
Fix the set of patients and their appointments $\{a_i\}$, parameters of their
service distributions $\{\mu_i\}$. Let $X(l)$ be the random total waiting time when we
use lead time $l$. The randomness of the system comes from the service
durations. We want to prove that $E[X(l)]$ is continuous w.r.t. $l$.

Let $W_i(l)$ is the random waiting time for patient $i$ when lead
time is $l$. Since $X(l)$ is just sum of $W_i(l)$, we only need to prove
$E[W_i(l)]$ is continuous w.r.t. $l$.

Now consider two lead times $l$ and $l+\epsilon$ ($\epsilon > 0$).
For any patient $i$ with appointment time $a_p$, we consider the
time window $[a_i - l - \epsilon, a_i - l]$. We call all these
windows \textit{critical windows}. Let $A_i$ be the event that
there are no patients got finished during \textit{critical window} of patient $i$.
Let $A = \cup_i A_i$ and suppose none of $A_i$ happens.
Consider the moments when we make decision for patient $i$ in two systems
at time $a_i-l$ and $a_i - l - \epsilon$. We can inductively argue
that the two systems see exactly the same set of unfinished patients
(including in-progress and waiting patients and patients who are already
assigned but haven't arrived yet) at each site. Because exponential
distribution is memory-less, the two system will calculate exactly
the same expected finish time for each site, and thus, assign patient $i$
to the same site. This shows that the end result will be the same
if there is no patient completion in \textit{critical windows}.

Let $\mu_{max}$ be the largest $\mu_i$. We can bound the probability
of $A_i$ by,
\[  k \mu_{max} \epsilon  \]
since there are at most $k$ patients being processed in parallel at any time
and the maximum rate of patient completion is $\mu_{max}$. By union bound,
we can bound the probability of $A$ by $n k \mu_{max} \epsilon$.

Now, we can prove $E[W_i(l)]$ is right-continuous
\begin{align*}
E[W_i(l+\epsilon)] - E[W_i(l)] &= E[(W_i(l+\epsilon) - W_i(l))I(A)] +
    E[(W_i(l+\epsilon) - W_i(l))(1 - I(A))] \\
 &= E[(W_i(l+\epsilon) - W_i(l))I(A)] \\
 &\le E[(|W_i(l+\epsilon)| + |W_i(l)|)I(A)]
\end{align*}
Ordering all patients by their appointment time, we can bound the
waiting time of patient $i$ by the sum of processing time of all
patients before him/her.
\[ W_i \le \sum_{j=1}^{i-1} d_j \]
This is true for any lead time. So
\[  E[W_i(l+\epsilon)] - E[W_i(l)] \le E[2(\sum_{j=1}^{i-1} d_j) I(A)] \le 2(i-1)\mu_{max} P[A]  \]
When $\epsilon \rightarrow 0^+$, we see that $P[A] \rightarrow 0$ and
thus $E[W_i(l)]$ is right continuous w.r.t. $l$. With similar argument,
we can show it's also left-continuous. This conclude our proof that
$E[X(l)]$ is continuous w.r.t. $l$.
\end{proof}
