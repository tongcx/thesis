\section{Continuity of system performance with respect to lead time}

In the experiment, we considered two lead times. It is an interesting question
to ask whether the system's performance changes continuously with respect to
the lead time. Here we give a proof for the following similar but simplified setting:
\begin{itemize}
\item There are $k$ sites, each with one machine.
\item There are $n$ patients. Patient $i$ has appointment time $a_i$
      and will arrive punctually.
\item The service duration $d_i$ of patient $i$ follows exponential
      distribution, possibly with different means.
\end{itemize}
For technical reason, we will first slightly modify our instance:
we will use a new appointment time $\tilde a_i = a_i + \delta u_i$,
where $\delta$ is a small constant and each $u_i$ is a uniform
random variable on $[-1,1]$. Now, all of the appointment times
are different almost surely.

The system will follow join-shortest-queue policy with lead time $l$.
Consider the decision epoch at time $\tilde a_i - l$, when the
system is deciding which site to assign patient $i$.
Let $P_{ij}(l)$ be the set of patients that are assigned to site $j$,
but have not completed their scans.
The system will calculate, for each site $j$, the time $C_{ij}(l)$ to finish patients $P_{ij}(l)$,
assuming that the scan duration will take expected value. The system
will assign patient $i$ to the site with earliest $C_{ij}(l)$.
If there is a tie, the patient will be assigned to the site with
lower number.

Let sample path $\omega$ includes all of the randomness of $u_i$ and $d_i$,
and let $X(l)$ be the random total waiting time of all of the patients.
We will prove the following theorem.

\begin{thm}
  $\E[X(l)]$ is continuous with respect to lead time $l \ge 0$.
\end{thm}

\begin{proof}
We want to prove $\E[X(l+\epsilon)] \rightarrow \E[X(l)]$ when
$\epsilon \rightarrow 0$.
Number all of the patients from $1$ to $n$ in non-decreasing order of $\tilde a_i$.
Note that the waiting time for any patient $i$ can be bounded
by the sum of scan duration of previous patients $\sum_{i' < i} d_{i'}$.
This implies that the total waiting time $X(l+\epsilon)$
can be bounded by $\sum_{i'=1}^n nd_{i'}$, which is
a integrable random variable. Thus, if we can prove
$X(l+\epsilon) \rightarrow X(l)$ almost surely
as $\epsilon \rightarrow 0$, the theorem will follow from
Dominated Convergence Theorem.

Now fix a sample path $\omega$ and the lead time $l$.
Consider the decision epoch for patient $i$.
Unless there are two sites with empty assigned patients $P_{ij}$, there will almost surely
be a non-zero gap between the earliest completion time $C_{ij}(l)$ and the rest.
Let $g_1>0$ be the smallest such non-zero gap among all of the patients.

Consider the closest pair of decision epoch and moment when a scan
is finished in time, and let $g_2$ be the time between them.
Almost surely, $g_2 > 0$.

We claim that $X(l+\epsilon)=X(l)$, if $|\epsilon| < \min\{g_1, g_2\}$.
Actually, we will prove the two systems will evolve exactly the same.
Suppose that on the contrary, the two systems diverge when
making decisions for patient $i$. Since $|\epsilon| < g_2$,
there is no scan completed between $\tilde a_i -l$ and
$\tilde a_i-\epsilon$, thus $P_{ij}(l) = P_{ij}(l+\epsilon)$ for any site $j$.
In other words, the two systems see the same set of assigned patients.
If there is some site with no assigned patients $P_{ij}(l)$,
both systems will choose such a site with lowest number.
Otherwise, because $|\epsilon| < g_1$, the site with lowest $C_{ij}(l)$
still has lowest $C_{ij}(l+\epsilon)$. The two systems still
make the same decision.

Thus, we have proved that the two system evolve
exactly the same when $\epsilon$ is sufficiently small.
This implies that $X(l+\epsilon) \rightarrow X(l)$ almost surely
when $\epsilon \rightarrow 0$, which completes our proof.
\end{proof}
