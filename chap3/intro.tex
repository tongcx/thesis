\section{Introduction}

Waiting time has been increasingly a problem in health-care industry. In May 2014, the Centers for Disease Control and Prevention reported average emergency department wait times is about 30 minutes. This phenomena is wide-spreading in all health-care industry: doctor office, emergent department, imaging facilities and other medical units all face this problem. Interestingly, most patients are so accustomed to this phenomena that they general expect waiting when visiting a doctor. More broadly, people are becoming more and more aware about the quality of services (waiting time is an important one) they receive. However, it seems hard to improve service quality for lots of cases.

There are two reasons that mainly account for this. First, health-care service usually consists of complicated process: patients need to be checked-in, triaged, do some lab testing, being seen by doctors, etc. If you look on the side of service provider, the picture is more complicated: patients information need to be managed, procedures need to coordinated, just to name a few. In each of these procedure, the may be some uncertainty: same procedure may takes random amount of time, there could be unexpected disrupts like machine failure, appointment cancellation, etc. All of this makes it very challenging to coordinate all procedures well and to have a consistent high service quality. The second reason is related to the fact that medical resources like doctor and machines are very expensive. In order to provide services to more patients, these resources need to be utilized very efficiently. This creates incentive for medical practitioners to make very tight schedules. Because there is limited leeway in the schedule, it's very easy for uncertainty and errors to propagate through the system and cause downstream impact. For example, patients may need to stay in waiting room for long time if something goes wrong for earlier patients.

One obvious way to improve is to bring in more medical resources. However, as we indicated earlier, this comes with significant costs. At 2015, health-care related spending accounts for 17.1\% of united states' annual GDP. It would be unsustainable to keep spending more and more. As a result, we need to search for alternative approaches where service quality can be improved without incurring too much additional cost.

Innovative technology has changed landscape of lots of traditional industry. For example, Google, Facebook, Uber, Airbnb fundmentally changed the way people search for information, contact family and friends, find a ride or seek short-term rental. It's becoming easier and easier to collect, transmit and analyze information. In health-care industry, information technology is also making great progress in making people's living easier and making cost lower. Facing with the challenge of improving service quality without bringing in much more very expensive medical resources, we may use information technology to identify ways to run the system more efficiently. Our work is in line with this thought.

In this chapter, we describe our work with New York Presbyterian hospital. Historically, some patients experienced significant waiting time for their outpatient MRI modality. The reason is mostly tight schedules due to high demand for MRI services. The scheduling is done by the call center of the hospital. There are about 20 people receiving calls at call center at the same time and they need to handle lots of other issuses beyond scheduling: check patients' information, verify their medical insurance, just to name a few. So it's quite hard to synchronize and make scheduling decisions together. As a result, they resort to simple rules put together by management team. Every scan is allocated a time slot of size multiple of 15min. Certain scans can only be conducted at certain time of the day due to nursing staffing constraints. Although the staffs make their best efforts to produce nice schedules, congestion may still appear now and then.

The hospital have three MRI facilities across Manhattan. This makes us to think whether we can exploit this fact to utilize resources more efficiently. In traditional queuing theory, we know that multiple-server queues perform better in terms of service quality, utilization compared with independent single-server queues. The reason is that customers can be served as soon as one resource become available and, thus, machines idle less. In our case, can we pool all three facilities together in some way to make it work more like a multiple-server queue? If achieved, this can increase the resources utilization and reduce patients waiting time at the same time without bringing in expensive new resource (a new MRI machines costs more than one million dollars). To overcome the problem that these facilities are not collocated, we use real-time information of each facility to make prediction into future. We speculate which facility is likely to run into congestion and which is not. We divert patients from congested site to ones that are not. In this way, we "pool" resources in all facilities together. Unbalance between schedules of all three sites may develop because of unexpected disruptions. We can discover that and smooth it out throughtout the day. Of course, patients may have personal constraints that they may not want to go to a different facility. We recognize this and give them the flexibility to decide whether or not to accept the diversion. We show that this approach can bring significant improvement on patient waiting time through simulation.

The most closely related literature to our problem is appointment scheduling. There are very good surveys on this topic \cite{gupta2008appointment, cayirli2003outpatient}. In that setting, usually it's assumed that the health-care providers already knows about patients that they need to see for a certain day together with all stochastic characteristics associated with each patient. The question is to decide the optimal schedule (appointment times of each patient) to balance patient waiting time, resource idle time and overtime. Several articles \cite{kaandorp2007optimal,denton2003sequential,begen2011appointment} have studied this problem and proposed algorithms optimal under certain assumptions. However, these results are not very applicable in our case. By assuming that health-care service provider can decide appointment time for patients, it makes these results mainly applicable to inpatients case where patients are not constrained their personal time constraints. For outpatient case, the appointment time is usually result of negotiation between patient's personal time constraint and medical resource's availability. Good planning is of course essential for smooth running of the system, but it would be better if we can be more robust to react to unexpected disruptions. This is not addressed by the appointment scheduling literature.
