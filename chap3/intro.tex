\section{Introduction}

New York Presbyterian (NYP) Hospital operates a large outpatient MRI service that
consists of three nearby sites within Manhattan, with 7 machines in total.
They provide services, to hundreds of types of scans and they process roughly 100
patients per day. However, patients sometimes experience long waiting time
-about 20\% of all patients wait more than 60 minutes. Furthermore, there are
still unfinished patients at the closing time of a site,
so sometimes the staffs there needs to work a significant amount of overtime. In this chapter,
we discuss a real-time queuing control policy to address these issues,
and a practical implementation to make this feasible, which takes advantage
of the modern-day reachabilities of patients just prior to their scheduled
appointment.

Patients need to make appointments in advance to reserve their slots.
The slot size, which is of length always an integer multiple of 15 minutes,
depends on the type of scan that the patient is scheduling.
Due to the high demand of MRI services, these machines often have tight
schedules. Together with the high variability in the procedures (which we
shall discuss in details below), these tight schedules cause long waiting times
and substantial overtime to occur.

There are several types of uncertainties that can cause reality to
deviate from the original schedule. There are some Same-Day Add-on Patients (SDAOP)
that will show up without appointments. These are mostly urgent cases
that need to be taken care of. There are also some same-day cancellations.

The most prevalent uncertainty is associated with scan duration and
patient arrival time. MRI scans usually consist of about a dozen of
image series, each with different scanning parameters. If the patient
being scanned moves too much during any series, the images produced
might be too blurred. In this case, the staff will have to redo this series.
Each series can take minutes, so redoing one will lengthen the scan
duration by a non-trivial amount. Furthermore, the scan procedure involves
many human factors. Patients might need extra time to prepare themselves
in the scan bed, and some even experience claustrophobia. Patients may need to
use the restroom during scans. Even the number of images to acquire depends
on the body size of the patients, and varies case by case.

Patient arrival time is another source of uncertainty. Although most
patients arrive on or before their appointment time, some may be
significantly late. This can be caused by bad traffic or the patient
simply leaves home late. Some patients need medical care while being
transported, and they will use Access-A-Ride which provides
such medical care. However, since Access-A-Ride accommodates
multiple patients per day and traffic in Manhattan is unpredictable,
patients might come to their appointments significantly early or late.

The staff usually refers to the situation of running behind schedule
as congestion. When significant congestion happens at one site, the staff there
will call other sites to see whether they can handle more patients. If so,
 they may choose to send patients to other sites using taxi. This practice
may help them balance loads among sites, but it also suffers from several
drawbacks. First, the decision is made solely based on current site status.
Sometimes, the other site is already congested by the time that the transported patient
arrives there. Second, after spending a long time waiting at one site, the patient
needs to be transported to another site. This brings inconvenience to both
patients and staff.

Inspired by their current practice, we developed a real-time queuing control
approach to pool resources across the three sites together. Instead of diverting
patients after congestion develops, we predict when and where congestion will
happen and react by diverting future patients to other sites. Frequently, this
can prevent congestion to happen. Also, since we are diverting future patients
who haven not shown up yet, they can go directly to the other site. This saves
time and effort for both patients and staff. We will describe our approach
in details and show computational results. In fact, our simulation result
based on 3 years of historical data shows that diverting just 3 patients per
day can lead to significant improvement in the number of patients experiencing
very long waiting time staff overtime.

The most closely related literature to our problem is in appointment scheduling.
There are excellent surveys on this topic \cite{gupta2008appointment, cayirli2003outpatient}.
In this setting, usually it is assumed that the health-care providers
already know about patients that must be seen on a particular day,
together with all stochastic characteristics associated with each patient.
The question is to decide the optimal schedule (appointment times of
each patient) to balance patient waiting time, resource idle time and
overtime. Several articles \cite{kaandorp2007optimal,denton2003sequential,begen2011appointment}
have studied this problem and proposed optimal algorithms under
certain assumptions. However, the results are not very applicable
in our case. The reason is that we are working with outpatient facility,
and we cannot really decide appointment times for patients. Nevertheless,
important insights can be drawn from the literature to help make
better schedules.
