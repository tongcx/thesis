\section{Introduction}

New York Presbyterian (NYP) hospital operates a big outpatient MRI service
consists of three nearby sites within Manhattan with 7 machines in total.
They provide services of hundreds types of scans and they process about 100
patients per day. However, patients sometimes experience long waiting time
(about 20\% patients wait more than 60min). Also, sometimes there are
still some unfinished patients at the closing time of some site,
and the staffs there need to work significant overtime. In this chapter,
we discuss a real-time queuing control policy to address these issues.

Patients need to make appointments in advance to reserve their slots.
The slot size (of 15min resolution) depends on the type of scan the patient is scheduling.
Due to high demand of MRI service, the machines often have tight
schedules. Together with high variability in the procedures (more on
shortly), they cause long waiting time and long overtime to happen.

There are several types of uncertainty that can cause reality to
deviate from schedule. There are some Same Day Add-on Patients (SDAOP)
that will show up without appointments. They're mostly urgent cases
that need to be taken care of. There are also some same-day cancellation.

The most prevalent uncertainty is associated with scan duration and
patient arrival time. MRI scans usually consist of about a dozen of
image series, each with different scanning parameters. If the patient
under scan moves too much during any series, the produced images
may be too blurred. In this case, the staff will have to redo this series.
Each series may take minutes, so redoing one will lengthen the scan
duration by non-trivial amount. Besides, the scan procedure involves
lots of human factors. Patients may need extra time to prepare themselves
in the scan bed, some even experience claustrophobia. They may need to
use restroom during scans. Even the number of images to acquire varies
case by case.

Patient arrival time is another source of uncertainty. Although most
patients arrive on or before their appointment time, some may be
significant late. This can be caused by bad traffic or the patient
simply leaves home late. Some patients need medical care during
transportation, so they will use Access-A-Ride which provides
such medical care. However, since Access-A-Ride accommodates
multiple patients per day and traffic in Manhattan is unpredictable,
patients may come to their appointments significantly early or late.

The staffs usually refer to the situation of running behind the schedule
as congestion. When significant congestion happens at one site, the staffs there
will call other sites to see whether they can handle more patients. If so,
 they may choose to send patients to other sites using taxi. This practice
may help them balance loads among sites, but it also suffers from several
drawbacks. First, the decision is made solely based on current site status.
Sometimes the other site is already congested by the time patients arrive
there. Second, after spending long time waiting at one site, the patients
need to transported to another site. This brings inconvenience to both
patients and staffs.

Inspired by their current practice, we developed a real-time queuing control
approach to pool resources across three sites together. Instead of diverting
patients after congestion develops, we predict when and where congestion will
happen and react by diverting future patients to other sites. Oftentimes, this
can prevent congestion to happen. Also, since we're diverting future patients
who haven't shown up yet. They can directly go to the other site. This saves
time and effort for both patients and staffs. We will describe our approach
in details and show computational results.

% admit machines are different, but
% admit that process time can vary

The most closely related literature to our problem is appointment scheduling.
There are excellent surveys on this topic \cite{gupta2008appointment, cayirli2003outpatient}.
In that setting, usually it's assumed that the health-care providers
already knows about patients that they need to see for a certain day
together with all stochastic characteristics associated with each patient.
The question is to decide the optimal schedule (appointment times of
each patient) to balance patient waiting time, resource idle time and
overtime. Several articles \cite{kaandorp2007optimal,denton2003sequential,begen2011appointment}
have studied this problem and proposed optimal algorithms under
certain assumptions. However, the results are not very applicable
in our case. The reason is that we're working with outpatients,
so we can't really decide appointment times for patients. Nevertheless,
important insights can be drawn from the literature to help make
better schedules.
