\subsubsection{Making decisions}

Throughout the day, we will try to make a few diversions. For each volunteer, we can try to divert him/her 1 hour before his/her appointment time. We call these times decision epochs. During each decision epoch, we can try all possible diversions (diverting patients to different site), and then look at which subset of them are valid. If there is more than one valid diversion, we will pick the one with the most improvement on overall cost (it beats the base scenario with most cases). The rationale here is that as long as we have enough confidence that diverted patient can wait less, then there is enough incentive for the volunteer to accept the diversion. Among these diversions, we want to choose the one that can improve overall system performance the best.

\paragraph{Remark} When making diversions, we're assuming there is no more diversions afterward. In other words, we're trying diverting patients one by one. There could be better arrangement by diverting several patients at the same time. However, the downside of this is that it's much more complicated. If we want to consider downstream impact of a diversion, we need to model this as stochastic dynamic programming. However this problem has very high dimensional state space so solving it is very impractical. More importantly, in real operation, it's not easy to coordinate several changes at the same time. The hospital people need to call all these patients simultaneously and the deal is sealed only if all of them agree to divert. 