\subsection{Dynamics and Events}

The whole system will evolve throughout the day and states of various entities will change. All of these are driven by events happening throughout the day. We will describe those events that drive the changes and the randomness associated with them.

\paragraph{Patient ready}

This event occurs when patients are ready for scan. After they come to the site assigned when scheduling the appointment, they will first do some preparation for the scan: fill paperworks, change clothes and maybe be put on intravenous therapy (IV). Patients are supposed to be ready at their appointment time, so they actually need to come earlier for preparation. After that, they will be seated at the waiting room until there is available machine to process him/her.

The first uncertainty is that the amount of time needed for preparation is uncertain. Some patients need more help on paperwork or maybe are slow in changing clothes. The second, and more important, uncertainty is their arrival time. Patients come with different transportation options. Some patients come by car. Manhattan is a place where bad traffic is very common. It's not hard to get stuck behind traffic for half a hour. Some patients come by subway or train. This is usually more predictable. However, if the patients are not familiar with these public transportation and, especially, how and where to transfer, then they may need additional time to get through the transportation.

Probably the most unpredictable transportation method is Access-A-Ride. Some patients have medical conditions that require them to have access to medical care throughout transportation. They may resort to Access-A-Ride. These vehicles have medical equipments onboard. However, each vehicle will serve multiple patients per day and appointment is needed. Drivers will optimize the route beforehand according to pickup and dropoff demands. However, if something goes wrong with earlier patients, then delay will propagate onto downstream patients. It's quite common that the driver cannot find a patient at the right time and location due to mis-communication or the car get stuck in traffic. As a result, patients using this service may arrive very early or late.

There is special type of patients called \textit{Same Day Add-on Patient}(SDAOP). They are emergent cases that will simply show up without appointments. We don't a prior know how many SDAOP will show up each day, when will they show up and what type of scans they need to do. Obviously, they will create congestion and all downstream patients will be pushed off.

\paragraph{Appointment cancellation}

Some patients with appointment may cancel their appointment by calling hospital. Their appointments will be removed from schedule. They may reschedule their appointment to another day. This will leave a hole in schedule and may cause machine to idle if no patients are there to fill the gap. But if the site is already running behind, this gap can help it to catch up with schedule and reduce waiting time of downstream patients. However, it's hard to predict who will cancel beforehand.

\paragraph{Scan begin}

Once a machine finish its current patient and become available, the next patient in waiting room will be called in for his/her scan. This is a scan begin event. We're assuming that we prioritize patients according to their appointment time, i.e. people with earlier appointment time will be processed first. This is not entirely true in reality, when a patient comes late and miss their appointment, the hospital may put him/her to wait and process other patients first. The rationale is that they don't want to make patients who haven't done anything wrong to wait. As a result, the late patient may need to wait for long time for a open slot. However, it's not easy to model this behaviour accurately. Thus, we decide to choose a simple and computationally efficient policy that's close enough to reality.

\paragraph{Scan completion} When patients finish their scans, there would be completion event. Patients will check out and leave the system and the machine will become available to process other another patient. The randomness introduced here is the scan duration and this is one of the most important randomness in the system.

There are several reasons why scan duration is highly variable. One most popular scan is BRAIN WITHOUT AND WITH CONTRAST. This involves scan patient's brain for two phases: one without contrast and the other with contrast. After the phase without contrast, a nurse will come to inject contrast via IV so that MRI machine can capture different set of images detailing certain structure in brain. The amount of time to inject contrast is uncertain, although probably not a very stochastic one.

Each type of scan consists of almost ten different image series, each with different parameters and different orientation. If the patient moves too much during any series, the images produced may become very blurred, rendering it useless for diagnosis. In this case, the technologist will have no choice but to redo this series. Each series may take minutes, so redoing one will introduce nontrivial delay to the system.

On top of that, patients may have different body size, which can also influence the amount of images needed to be acquired. Patients may need to go to restroom during the scan, they may be of poor health condition and require other people to help them move onto and off scan table. All these and other factors can affect amount of time the machine is occupied and they together make the scan duration highly variable. Sometimes, patients may experience claustrophobia during scan. The amount of time needed to allow patients to get used to the narrow space inside MRI machines and calm down is uncertain and also highly variable. This will increase duration this patient occupied the resource and add delay to the system.
