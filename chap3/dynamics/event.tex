\subsection{Dynamics and Events}

The whole system will evolve throughout the day and states of various entities will change. All of these are driven by events happening throughout the day. We will describe those events that drive the changes and the randomness associated with them.

\paragraph{Patient arrival} Patients will come to the facility that he/she is assigned to. However, patients may not come at their appointment time, they may come early or later. This will introduce some randomness to the system through uncertainty of patients' arrival time. After they come, they will be seated at the waiting room until there is available machine to process him/her. There is another type of patients called same day add-on patient (SDAOP). They are emergent cases that will simply show up without appointments. In other words, they will be surprises to our system since we don't a prior know about them and we don't have idea on when will they come and what type of scans they need to do.

\paragraph{Appointment cancellation} Some patients with appointment may cancel their appointment by calling hospital. This will result in
removing this patient from schedule. They may reschedule their appointment to some future day, but we don't need to consider this since it's going to be another day. Both who will cancel their appointments and when will they make the phone call are uncertain. Cancellation may result in resource idling if no patients are there to fill the gap. But on the other hand, if the site is already running behind schedule, cancellation will help it to catch up and may reducie the waiting time of downstream patients.

\paragraph{Scan begin} After patients come to the facility, they will be placed in waiting room. Once a machine become available, the next patient will be called in for his/her scan. This is a scan begin event. We're assuming that we prioritize patients according to their appointment time, i.e. people with earlier appointment time will be processed first. This is not entirely true in reality, when a patient comes late and miss their appointment, the hospital may put him/her to wait and process other patients first. The rationale is that they don't want to make patients who haven't done anything wrong to wait. As a result, the late patient may need to wait for long time for a open slot. However, it's not easy to model this behaviour accurately. Thus, we decide to choose a simple and computationally efficient policy that's close enough to reality.

\paragraph{Scan completion} When patients finish their scans, there would be completion event. In this case, patients will leave the system and the machine will become available to process other patients. The randomness introduced here is the scan duration. Scan duration may be the most important randomness in the system. It's highly variable, which may cause lots of congestion.

