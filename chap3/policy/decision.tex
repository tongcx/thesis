\subsection{Making decisions}

Throughout the day, we will try to make a few diversions. For each volunteer, we can try to divert him/her 1 hour before his/her appointment time. We call these times decision epochs. During each decision epoch, we can try all possible diversions for this patient each corresponding to one site, and then look at which subset of them are valid. If there is more than one valid diversion, we will pick the one with the highest probability on objective improvement. The rationale here is that as long as we have enough confidence that diverted patient wait less, then there is enough incentive for the volunteer to accept the diversion, so we should focus on overall system performance.

\paragraph{Remark} When making diversions, we're imagining there is no more diversions afterward. In other words, we're trying diverting patients one by one. There could be better arrangement by diverting several patients at the same time. However, the downside of this is that it's much more complicated. If we want to consider downstream impact of a diversion, we need to model this as stochastic dynamic programming. However this problem has very high dimensional state space so solving it is very impractical. More importantly, it's not easy to coordinate several changes at the same time. The hospital staffs need to call all these patients simultaneously and the deal is sealed only if all of them agree to divert. This may bring significant communication and coordination issues.
