\begin{abstract}
\noindent One method to obtain high-quality bid prices for network revenue management problems involves using the approximate linear programming approach on the dynamic programming formulation of the problem. This approach ends up with a linear program whose number of constraints increases exponentially with the number of flight legs in the airline network. The linear program is solved by using constraint generation, where each constraint can be generated by solving a separate integer program. The necessity to solve integer programs and the slow convergence behavior of constraint generation are generally recognized as drawbacks of this approach. In this paper, we show how to effectively eliminate these drawbacks. In particular, we establish that constraint generation can actually be carried out by solving minimum-cost network flow problems with natural integer solutions. Furthermore, using the structure of minimum-cost network flow problems, we a priori reduce the number of constraints in the linear program from exponential in the number of flight legs to linear. It turns out that the reduced linear program can be solved without using separate problems to generate constraints. The reduced linear program also provides a practically appealing interpretation.~Computational experiments indicate that our results can speed up the computation time for the approximate linear programming approach by a factor ranging between 13 and 135.



\end{abstract}

\begin{abstract}
\noindent We consider assortment optimization problems under the multinomial logit model, where the parameters of the choice model are random. The randomness in the choice model parameters is motivated by the fact that there are multiple customer segments, each with different preferences for the products, and the segment of each customer is unknown to the firm when the customer makes a purchase. This choice model is also called the mixture-of-logits model. The goal of the firm is to choose an assortment of products to offer that maximizes the expected revenue per customer, across all customer segments. We establish that the problem is NP-complete even when there are just two customer segments. Motivated by this complexity result, we focus on assortments consisting of products with the highest revenues, which we refer to as revenue-ordered assortments. We identify specially structured cases of the problem where revenue-ordered assortments are optimal. When the randomness in the choice model parameters does not follow a special structure, we derive tight approximation guarantees for revenue-ordered assortments.
We extend our model to the multi-period capacity allocation problem, and prove that, when restricted  to the revenue-ordered assortments,  the mixture-of-logits model possesses the nesting-by-fare-order property.  This result implies that \mbox{revenue-ordered} assortments can be incorporated into existing revenue management systems through nested protection levels. Numerical experiments show that revenue-ordered assortments perform remarkably well, generally yielding profits that are within a fraction of a percent of the~optimal.
\end{abstract}

