\section{Introduction}

Although algorithm design for the {\it facility location problem} and its many variants has been the focus of a significant body of research, there is an important class of generalizations that has received much attention, but only with more limited success. In the facility location problem, one has a set of demand points and a set of potential facility locations, with specified costs for opening, and the aim is to optimize the choice of open facilities so as to minimize the sum of opening costs and the cost of assigning each demand point to an open facility, where the assignment costs derive from an underlying metric space containing all of the locations. This model places no restrictions on the clients served by each facility, which is typically an unreasonable assumption, and so a great deal of work has been done to address further constrained models. The simplest extension is to impose a capacity constraint, but for a wide range of applications, it is more appropriate to distinguish types of service provided and required, and it is on this class of problems that we shall focus. Facility location models are closely linked to a number of inventory models, since one can view a facility as a point in time at which an order is placed, and the assignment cost then corresponds to the cost of filling a particular demand from that order, where one can have fixed ordering costs, unit ordering costs, and inventory holding costs easily incorporated into this setting. In this chapter, we shall consider both facility location and inventory management problems that arise in the setting with type constraints. 

Since the facility location problem is NP-hard, much of the algorithmic work in this domain has focused on the design of approximation algorithms, and it has proved to be a fertile ground for the development of many of the now-standard techniques in algorithm design:
researchers have applied deterministic and randomized rounding \cite{shmoys1997approximation,li20131,chudak2003improved}, primal-dual methods \cite{jain2001approximation}, local search\cite{korupolu2000analysis}, and even greedy-type algorithms \cite{jain2003greedy} to this problem. For the classical facility location problem, the best approximation guarantee currently known is a 1.488-approximation algorithm (where a {\it $\rho$-approximation algorithm} is a polynomial-time algorithm that is guaranteed to find a feasible solution of objective function value within a factor of $\rho$ of the optimum), which was obtained by randomized rounding \cite{li20131}.
For the {\it $k$-median problem}, in which there are no facility costs but one is limited to opening only $k$ facilities, strong results are known via a range of techniques as well \cite{li2013approximating,jain2001approximation,arya2004local}, and analogous results (though more limited) are known for the capacitated version of the facility location problem \cite{chudak2005improved,an2014lp,pal2001facility,bansal20125}. One particularly notable open problem is to derive good algorithms for the setting in which the opening/ordering costs are submodular set functions of the set of demand points assigned (or for suitably general special cases).

The research presented here aims to derive results in which each facility may have a
different serving capability. In the most general case along these lines, for each client,
we can specify the set of facilities that can serve it. However, it is easy to reduce the
set cover problem to it in an approximation-preserving manner (see appendix); thus, approximation
hardness results known for the set cover problem would apply (and hence no sub-logarithmic
guarantee is possible unless P=NP). Hence, the issue is to model sufficiently robust
classes of problems that capture interesting application settings, and yet by-pass this
hardness result.  We consider the following classes of problems, in which  each facility
and client is assigned a type, and there is a specified partial ordering among types so
that a client must be served by any open facility of equal or higher type.  
The structure of the partial ordering will determine the nature of our problems. 
With a general partial order, one can again reduce set cover to the problem; we consider
two structured variants to bypass this. 

\begin{itemize}
\item \textbf{Facility Location with Hierarchical Types.} In this problem, the types form a rooted tree in which a facility of a given type can serve any client that has a type that is a descendant of the facility type in the tree (where a node is trivially a descendant of itself).

\item \textbf{Pepsi-Coke Facility Location.} In this problem, there are only two types of facilities: Pepsi facilities and Coke facilities. There are three types of clients: Pepsi clients, Coke clients and Anything-will-do clients. As the names suggest, Anything-will-do clients can be served by any facility, but the other clients can only be served by facilities of the same type. The partial ordering here is a ``V-shape,'' where there are no facilities of the bottom type.

\end{itemize}

A model somewhat weaker than the former setting was considered by Barman and
Chawla~\cite{DBLP:journals/corr/abs-1110-4150}, who introduced a problem called the {\it
  redundancy-aware facility location problem}, where each client needs a set of services
and all of these services need to be installed at the same facility location in order to
serve that client; they gave an LP rounding algorithm with approximation guarantee of 27
for the case that all clients' demanded service sets are laminar. The facility location
problem with hierarchical types is the generalization of their problem if we group
services belong to the same service set to a single facility. We include more details on the relationship in appendix. In contrast, the first main
result of this chapter is a simple primal-dual 3-approximation algorithm for the facility
location problem with hierarchical types. We show how to add a simple prioritization rule
to the pruning phase of the algorithm of Jain and Vazirani \cite{jain2001approximation}
leads to this result. Furthermore, the analysis gives a stronger result; we show that it
is a Lagrangian-multiplier-preserving algorithm, in that not only is the sum of the
assignment and facility costs at most three times the optimum, but the sum of the
assignment and 3 times the facility cost is at most 3 times the optimal cost. We are
unable to convert this to an analogous $k$-median result, but we can augment this analysis with a rescaling technique \mbox{to obtain a 1.85-approximation algorithm.}

For the Pepsi-Coke facility location problem, we add a new dimension to the by-now standard LP rounding-by-clustering techniques that arises due to the fact that when choosing a cluster for one of the Anything-will-do clients $j$, a facility used by the LP solution for that client, cannot necessarily serve all of the clients served by any facility fractionally serving $j$. We overcome this difficulty by separately performing two clustering constructions,  but then couple the outcomes by relying on an auxiliary matroid intersection computation, where a feasible fractional solution to the matroid intersection instance is derived from the original fractional solution (in a manner similar to a technique used by Swamy \cite{swamy2013improved}). Consequently, we obtain a 3-approximation algorithm for the Pepsi-Coke facility location problem.

Analogous to the development of approximation algorithms for facility location problems,
there has been a corresponding thread of research investigating results of a variety of
inventory management problems, primarily the {\it lot-sizing} and the {\it joint
  replenishment
  problem}\cite{levi2008approximation,levi2008constant,levi2006primal,ApproxjrpD:Beinkowski,cheung2014submodular}. In
some elements, these problems are sometimes simpler than their facility location
equivalents, since the assignment costs, in addition to obeying the triangle inequality,
often have an even more refined structure due to the linear nature of ordering/demand time
periods. However, these problems have a harder element, since the metric is definitely not
symmetric - one cannot serve a demand point earlier in time than the ordering
period. Hence, it is natural to ask for the analogs of the two results discussed above. Surprisingly, for the lot-sizing problem with hierarchical types, we show that the problem can be solved dynamic programming in polynomial time, in a manner exactly analogous to a recent result \cite{cheung2014submodular}. However, unlike for the unconstrained lot-sizing problem, for which the natural linear programming relaxation is the basis for a primal-dual polynomial-time algorithm as well \cite{levi2006primal},
we give an example for which there is an integrality gap of 4/3 for the lot-sizing problem with hierarchical types.
