\section{Preliminaries}

In this section, we introduce some definitions and notation to be used throughout this chapter.
%
Let $\mathcal{F}$ be the set of potential facilities; opening facility $i \in \mathcal{F}$ has associated non-negative cost $f_i$.
Let $\mathcal{D}$ be the set of demand points. We need to assign each client $j \in \mathcal{D}$ to some open facility $i$ and it costs $c_{ij}$.
%
In addition, there is set of types $\mathcal{T}$, forming a partial ordering. Each facility $i$ has a type $t(i)$ and each client $j$ has type $t(j)$. Facility $i$ is capable of serving client $j$ if and only if $t(j) \le t(i)$ in the partial ordering. Types indicate the serving capabilities. Denote $\mathcal{F}(j)$ be all facilities that can serve client $j$.

Suppose we open the set of facilities $S$ in the solution and assign client $j$ to $\sigma(j)$. Then the total cost of this solution is
$\sum_{i \in S} f_i + \sum_{j \in \mathcal{D}} c_{\sigma(j)j}$, i.e., 
the sum of total facility opening cost and total assignment cost. The goal is to find a feasible solution with minimum total cost.

Our algorithm and analysis will rely on the following primal linear-programming (LP)
relaxation \eqref{primalLP} and its dual \eqref{dualLP}. %and dual linear programs (LPs). 
%which is described in following.

%\vspace*{-1ex}
\noindent \hspace*{-4ex}
\begin{minipage}[t]{.53\linewidth}
\begin{align*}
\min \quad & \sum_{i \in \mathcal{F}} f_i y_i + \sum_{j \in \mathcal{D}}  
\sum_{i \in \mathcal{F}(j)} c_{ij}x_{ij}  \tag{P} \label{primalLP} \\
\text{s.t.} \quad & \sum_{i \in \mathcal{F}(j)} x_{ij} \ge 1 \quad \forall {j \in \mathcal{D}} \\
& x_{ij} \le y_i    \quad \forall j \in \mathcal{D}, i \in \mathcal{F}(j)   \\
& x_{ij}, y_i \ge 0 \quad \forall {i \in \mathcal{F}}, j\in\mathcal{D}    %\\
%& x_{ij} \ge 0 \quad &&\forall j \in \mathcal{D}, i \in \mathcal{F}(j)
\end{align*}
\end{minipage}
\quad
\begin{minipage}[t]{.5\linewidth}
\begin{align*}
\max \quad & \sum_{j \in \mathcal{D}} v_j \tag{D} \label{dualLP} \\
\text{s.t.} \quad & v_j \le c_{ij} + w_{ij} \quad \forall j \in \mathcal{D},i \in \mathcal{F}(j)\\
& \sum_{j: i \in \mathcal{F}(j)} w_{ij} \le f_i \quad \forall i \in \mathcal{F} \\
& v_j, w_{ij} \ge 0  \quad \forall j \in \mathcal{D}, i \in \mathcal{F}(j). %\\
%& v_j \ge 0 \quad \forall j \in \mathcal{D}
\end{align*}
\end{minipage}

\medskip

%\begin{align}
%\min \quad & \sum_{i \in \mathcal{F}} f_i y_i + \sum_{j \in \mathcal{D}} \sum_{i \in
%  \mathcal{F}(j)} c_{ij}x_{ij} \tag{P} \label{LP:main} \\
%\text{s.t.} \quad & \sum_{i \in \mathcal{F}(j)} x_{ij} \ge 1 \quad \forall {j \in \mathcal{D}} \label{con:P:demand\\
%& x_{ij} \le y_i    \quad \forall j \in \mathcal{D}, i \in \mathcal{F}(j)   \label{con:P:reliance}\\
%& y_i \ge 0 \quad \forall {i \in \mathcal{F}}   \label{con:P:ynonneg}\\
%& x_{ij} \ge 0 \quad \forall j \in \mathcal{D}, i \in \mathcal{F}(j) \label{con:P:xnonneg}
%\end{align}

Any binary feasible solution to \eqref{primalLP} corresponds to a feasible solution. $y_i
= 1$ indicates that facility $i$ is open and $x_{ij} = 1$ indicates that client $j$ is
served by facility $i$. 

%Taking the dual, we get the following 
%
%\begin{align}
%\max \quad & \sum_{j \in \mathcal{D}} v_j \tag{D} \label{dualLP} \\
%\text{s.t.} \quad & v_j \le c_{ij} + w_{ij} \quad \forall j \in \mathcal{D}, i \in \mathcal{F}(j)\\
%& \sum_{j: i \in \mathcal{F}(j)} w_{ij} \le f_i \quad \forall i \in \mathcal{F} \\
%& w_{ij} \ge 0  \quad \forall j \in \mathcal{D}, i \in \mathcal{F}(j) \label{con:facility} \\
%& v_j \ge 0 \quad \forall j \in \mathcal{D}
%\end{align}

For the dual \eqref{dualLP}, we can intuitively think $v_j$ as budget of client $j$ and $w_{ij}$ as its contribution towards opening facility $i$. The first set of constraints say that the budget can cover both assignment cost and contribution toward opening cost. The second set of constraints say that the total contribution toward one facility cannot exceed its opening cost.
