\section{The Pepsi-Coke facility location problem}

In the Pepsi-Coke problem, there are only two types of facilities: Pepsi facilities and Coke facilities. There are three types of clients: Pepsi clients $\mathcal{D}_P$, Coke clients $\mathcal{D}_C$ and Anything-will-do clients $\mathcal{D}_A$. As the names suggest, Anything-will-do clients can be served by any facility but the other clients can be served only by facilities of the same type. The type tree is a V shape where no facilities are of the bottom type. The assignment cost $c_{ij}$ is also a symmetric metric.

We present a randomized rounding 3-approximation algorithm here. The main difficulty is that we may open a Pepsi facility for a cluster centered at an Anything-will-do client, and then we cannot serve any Coke facilities in the same cluster. We show how to overcome this difficulty by doing two clusterings and then coupling them using matroid intersection.

\subsection{The algorithm}

The algorithm is based on clustering and randomized rounding. The basic idea is to construct several clusters, where each cluster has one center client and several facilities. Each non-center client is assigned to some cluster. Then we can open one facility in each cluster and assign all clients in that cluster to it. You can find a detailed explanation in \cite{williamson2011design} Section 5.8.

However, in our case, we need to deal with clients of different types. Naive clustering will leave us with some non-center client not being able to be served via cluster center. We need to do something more delicate.

Now, we describe the algorithms. First, solve the linear programming and get optimal solutions $x^*, y^*, w^*, v^*$. Assume here that $x^*_{ij}$ is either $y^*_i$ or $0$; we will need this later. This can be achieved by making several copies of facilities as in \cite{chudak2003improved}.

\begin{defn}
We say client $j$ {\it neighbors} facility $i$ if $x^*_{ij} > 0$.
Denote the {\it neighborhood} of client $j$ to be $N(j) = \{i \in \mathcal{F}(j) : x^*_{ij} > 0 \}$.
\end{defn}

\noindent
Let $C^*(j) = \sum_{i \in \mathcal{F}(j)} x^*_{ij}c_{ij}$
be the fractional assignment cost of client $j$.

\textbf{Clustering} The purpose of clustering is to identify a set of center clients and ensure each non-center client is close to some center client. We will do two clusterings here: one for all Pepsi-only and Coke-only clients and another
for Any-will-do clients.

Let $D_1$ be empty in the beginning. We consider client $j \in \mathcal{D}_P \cup \mathcal{D}_C$
one by one in ascending order by $v^*_j + C^*(j)$. If $N(j)$ does not intersect $N(k)$
for any $k \in D_1$, we add $j$ to $D_1$. After we considered all clients in $\mathcal{D}_P \cup \mathcal{D}_C$,
we will have our center-clients for the first clustering round.

Let $D_2$ be empty. Then we consider client $j \in \mathcal{D}_A$ one by one
in ascending order by $v^*_j + C^*(j)$.
If $N(j)$ does not intersect with $N(k)$ for any $k \in D_2$, we add $j$ to $D_2$.
In the end, we will have our center-clients for the second clustering round.

\begin{fact}
Let $D = D_1 \cup D_2$ be the set of center clients from both clustering rounds.
For any $j \in D$, $y^*(N(j)) = 1$.
\end{fact}

\textbf{Rounding} We would like to open exactly one facility for each cluster. However, since we have two sets of clusters and there could be some client $j \in D_1$ and $k \in D_2$ sharing the same facility in their neighborhoods, we cannot just randomly open one facility per cluster. If we use $z$ to denote our decision of which facility to open, meaning that $z_i = 1$ if we want to open facility $i$ and $z_i = 0$ otherwise. Then our requirement of opening exactly one facility in each cluster can be expressed as

%\begin {align}\
\[
\sum_{i \in N(j)} z_i = 1 \quad \forall j \in D_1, \qquad %\\
\sum_{i \in N(j)} z_i = 1 \quad \forall j \in D_2, \qquad %\\
z_i \ge 0 \quad \forall i \in \mathcal{F}.
\]
%\end{align}

Note that, since $N(j)$'s are disjoint within $D_1$ or $D_2$, we essentially have
a matroid intersection base polytope for two partition matroids.
We actually know one fractional point within this polytope: $y^*$.

From \cite{grotschel1981ellipsoid}, we know that we can decompose,
in polynomial time, $z^*$ into a convex combination of polynomial number of vertice of
the matroid intersection polytope:s
$z^* = \sum_{v} \lambda_v z_v$.
Here $\{\lambda_v\}$ is a distribution over vertices $\{z_v\}$.
In addition, we know the matroid intersection polytope is integral \cite{schrijver2003combinatorial}, and
so each $z_v$ is binary. Thus, each $z_v$ corresponds to a set of facilities to open.

We can sample vertex $z_v$ according to distribution $\{\lambda_v\}$
and then open the set of facilities corresponding to $z_v$.
In this way, we make sure that for each center client $j \in D$, we opened exactly one facility in $N(j)$.
This rounding method is inspired by \cite{swamy2013improved}.

\textbf{Assignment} We describe here one assignment that we will show to be of low cost later.
For each center-client $j$, we know exactly one facility $i \in N(j)$
is opened and we will assign $j$ to $i$.
For any non-center client $j$, it was classified as non-center during one of the clustering phases
because, at the time when we consider $j$, some center client $k$ has neighborhood $N(k)$
intersecting $j$'s neighborhood $N(j)$. We will assign $j$ to the same facility that $k$ was assigned to.

Here we will show that this assignment is feasible.
For anything-will-do clients, they are fine as long as they are served by some facility, so we do not need to worry about them.
For a Pepsi-only client $j$, consider the center client $k$ it was assigned to.
Client $k$ must be a Pepsi-only client. Because we separated the clustering phase of anything-will-do clients
and other clients, so $k$ must be a Pepsi-only client or a Coke-only client.
But it cannot be Coke-only, because otherwise $N(k)$ are all Coke facilities and it will not intersect with
$N(j)$ which consists of only Pepsi facilities.
As a result, we know $k$ is assigned to an open Pepsi facility which can also serve $j$.
The same argument will work for a Coke-only client.


\subsection{The analysis}

Now, we can analyze the total cost of our algorithm. Since we open facilities $z_v$ with distribution $\lambda_v$, 

\begin{lem}
The expected facility opening cost is exactly %facility opening cost
$\sum_{i \in \mathcal{F}} f_i y^*_i$.
\end{lem}

Now let us consider assignment cost. For any center-client $j$,
we know that $i \in N(j)$ is opened with probability $y^*_i = x^*_{ij}$,
so the expected assignment cost of $j$ is exactly $C^*(j)$. For a non-center client, we have this lemma.

\begin{lem}
The expected assignment cost of a non-center client $j$ is at most
$2 v^*_j + C^*(j)$.
\end{lem}
\begin{proof}
Take any non-center client $j$; from the last section, we know that there is a center client $k$
that prevented us from making $j$ a center. This means that there is a facility $i \in N(j) \cap N(k)$.
Suppose that during the rounding phase, we opened facility $\ell \in N(k)$. Then we will assign $j$ to
$\ell$. 
%and let's bound its cost,
We have $c_{j\ell} \le c_{ij} + c_{ik} + c_{\ell k}$.
By complementary slackness of optimal LP solution, we know that $c_{ij} \le v^*_j, c_{ik}\le v^*_k$, so
$c_{j\ell } \le v^*_j + v^*_k + c_{\ell k}$.
Because $c_{\ell k}$ has expectation $C^*(k)$, we know
$E [c_{j\ell }] \le v^*_j + v^*_k + E [c_{\ell k}] = v^*_j + v^*_k + C^*(k)$.
Since we considered $k$ before $j$, we have $v^*_k + C^*(k) \le v^*_j + C^*(j)$, thus,
$E [c_{j\ell}] \le 2v^*_j + C^*(j)$.
\end{proof}

With all this, we know the total expected cost is at most
\[  \sum_{i \in \mathcal{F}} f_i y^*_i + \sum_{j \in \mathcal{D}} C^*(j) + \sum_{j \not \in D} 2v^*_j
\le 3 OPT .   \]

Note that instead of randomly sampling a facility to open, we can try each $z_v$ and pick the one
with lowest total cost. This will guarantee to no more than the expected total cost.
In other words, this derandomizes our algorithm and proves the main theorem.

\begin{thm}
There is a deterministic polynomial 3-approximation algorithm for Pepsi-Coke facility location problem.
\end{thm}
